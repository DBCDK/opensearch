\documentclass{article}
\usepackage{a4wide}
\usepackage[utf8]{inputenc} %             be able to use danish letters
\usepackage[danish]{babel} %                and danish macros
\usepackage{url}                            %, hyperref} hyperref makes clickable refs.
\usepackage{amsmath}
\renewcommand\sfdefault{phv}%               use helvetica for sans serif
\renewcommand\familydefault{\sfdefault}%    use sans serif by default
\setlength{\parskip}{2mm plus1mm minus1mm}% a bit of (rubber) space between paragraphs

\author{Lars Vensild Hørnell \and Søren Mollerup \and Steen
  Manniche\thanks{\{lvh, shm, stm\}\@dbc.dk}}
\date{\today}
\title{Usecase: Datadock, aflevering af filpeger til kø \thanks{Dette dokument kan hentes fra \texttt{svn://svn.dbc.dk/OpenSearch/project\_information}}}

\begin{document}

\maketitle

\newpage

\tableofcontents
% Det er hensigten, at alle informationer der er i < eller > tegn skal
% slettes inden dokumentet bygges

\section{Usecase beskrivelse}

Efter at datadock har lagt dataobjektet i Fedora repositoriet giver
det en peger til dataobjektet til Processeringskøen. 

\section{Type}
Konkret


\section{Relationer}
<Om use casen f.eks. anvender andre use cases>


\section{Aktører}

\begin{itemize}
\item Datadock
\item Fedora repositoriet
\item Processeringkøen
\end{itemize}

\section{Forudsætninger}
Dataobjektet er valideret, lagt i repositoriet og et tidsestimat er
foretaget. 


\section{Initiering}
Datadock har modtaget en filpeger fra Fedora repositoriet.


\section{Beskrivelse}
Datadock lægger filpegeren fra Fedora i Processeringskøen. Når dette
er gjort, er Datadock færdig med behandling af pågældende objekt. Hvis
det ikke lykkeds at lægge filpegeren på køen smider Datadock
[UnableToQueueException].  


\section{Terminationstilstande}

\subsection{Succes}
Filpegeren er lagt på Processeringskøen.

\subsection{Exception}
\textbf{UnableToQueueException} smides når Datadock ikke kan lægge
filpegeren fra repositoriet på Processeringskøen. Den information der
følger med undtagelsen skal indeholde en mulig fejlmeddelse fra
Processeringskøen og filpegeren. 


\section{Status på usecase}
Hvordan en [UnableToQueueException] håndteres og af hvem er ikke afklaret.

\end{document}
