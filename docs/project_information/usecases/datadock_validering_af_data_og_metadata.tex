\documentclass{article}
\usepackage{a4wide}
\usepackage[utf8]{inputenc} %             be able to use danish letters
\usepackage[danish]{babel} %                and danish macros
\usepackage{url}                            %, hyperref} hyperref makes clickable refs.
\usepackage{amsmath}
\renewcommand\sfdefault{phv}%               use helvetica for sans serif
\renewcommand\familydefault{\sfdefault}%    use sans serif by default
\setlength{\parskip}{2mm plus1mm minus1mm}% a bit of (rubber) space between paragraphs

\author{Lars Vensild Hørnell \and Søren Mollerup \and Steen
  Manniche\thanks{\{lvh, shm, stm\}\@dbc.dk}}
\date{\today}
\title{Use case: datadock- Validering af data og metadata\thanks{Dette dokument kan hentes fra \texttt{svn://svn.dbc.dk/OpenSearch/project\_information}}}

\begin{document}

\maketitle

\newpage

\tableofcontents
% Det er hensigten, at alle informationer der er i < eller > tegn skal
% slettes inden dokumentet bygges

\section{Use case for validering af indkommen data og metadata}

\subsection{Use case beskrivelse}
Usecase, der beskriver hvorledes validering af indkommen data og metadata skal foregå.
Datadocken validerer at de nødvendige metadata er tilstede, samt at dataformatet er understøttet.

\section{Type}
Konkret

\section{Relationer}

\section{Aktører}
\begin{itemize}
\item NEP
\item Processkø
\end{itemize}


\section{Forudsætninger}
For at starte valideringen af data samt metadata, er det eneste krav
til datadocken, at den er klar til at påbegynde en ny opgave, dvs. at
tidligere opgaver er afsluttet.

\section{Initiering}
Datadockens validering af data samt metadata, startes når en der
kommer en datastrøm fra NEP'en.

\section{Beskrivelse}
Datadock bliver givet en datastrøm fra NEP'en.  Herefter validerer
Datadock data samt metadata, efter nedenstående regelsæt (beskrevet
med flowchartet "Datadock - validation of incoming data"):

\begin{itemize}
\item Er metadata tilstede ?
  smider [NeededDataNotFoundException]
\item Er data tilstede ?
  smider [NeededDataNotFoundException]
\item Er indsenderen kendt ?
  smider [NotAuthorizedException]
\item er MIME-typen kendt ?
  smider [UnknownFormatException]
\item Er (natur)sproget kendt ?
  smider [UnknownFormatException]
\item Findes der en passende analyzer ?
  smider [UnknownFormatException]
\end{itemize}

Hvis data samt metadata bliver valideret succesfuldt efter ovenstående kriterier, bliver der foretaget en estimerinng af tidsforbrug, og data samt metadata bliver indlæst i repositoriet, og en filpeger til repositoriet bliver lagt på processkøen.

\section{Terminationstilstande}

\subsection{Succes}
Dataemnet er indlæst i repositoriet, og den tilhørende filpeger er lagt på processkøen.
Brugeren får returneret tidsestimatet.


\subsection{Exception}

\begin{itemize}
\item \textbf{NeededDataNotFoundException}\newline Datadock kan ikke
  finde data eller metadata eller begge dele. Denne exception bliver
  gennem receptionslaget sendt tilbage til brugeren, i en form der
  bestemmes af receptionslaget.

\item \textbf{NotAuthorizedException}\newline
  Datadock er kan ikke identificere indsenderen. Denne exception bliver
  gennem receptionslaget sendt tilbage til brugeren, i en form der
  bestemmes af receptionslaget.

\item \textbf{UnknownFormatException}\newline Datadock har ikke
  kendskab til MIME-typen, eller også er (natur)sproget ukendt, eller
  der findes ingen analyzer til pågælende sprog. Denne exception bliver
  gennem receptionslaget sendt tilbage til brugeren, i en form der
  bestemmes af receptionslaget.
elelse til Servicelaget.
\end{itemize}

\section{Status på use case}

Vi mangler en nærmere beskrivelse af hvordan receptionslaget interagere med NEP'en.
\end{document}
