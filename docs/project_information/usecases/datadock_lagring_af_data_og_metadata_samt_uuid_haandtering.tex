\documentclass{article}
\usepackage{a4wide}
\usepackage[utf8]{inputenc} %             be able to use danish letters
\usepackage[danish]{babel} %                and danish macros
\usepackage{url}                           %, hyperref} hyperref makes clickable refs.
\usepackage{amsmath}
\renewcommand\sfdefault{phv}%               use helvetica for sans serif
\renewcommand\familydefault{\sfdefault}%    use sans serif by default
\setlength{\parskip}{2mm plus1mm minus1mm}% a bit of (rubber) space between paragraphs

\author{Lars Vensild Hørnell \and Søren Mollerup \and Steen Manniche\thanks{\{lvh, shm, stm\}\@dbc.dk}}
\date{\today}
\title{Usecase: datadock- Lagring af data og metadata samt uuid håndtering\thanks{Dette dokument kan hentes fra \texttt{svn://svn.dbc.dk/OpenSearch/project\_information}}}

\begin{document}

\maketitle

\newpage

\tableofcontents
% Det er hensigten, at alle informationer der er i < eller > tegn skal
% slettes inden dokumentet bygges

\section{Use case for lagring af data og metadata samt uuid håndtering}

\subsection{Use case beskrivelse}
Usecase, der beskriver hvorledes et valideret og estimeret dataemne
bliver lagret i repositoriet, samt hvorledes et uuid bliver genereret og håndteret.

\section{Type}
Konkret

\section{Relationer}

\section{Aktører}
\begin{itemize}
\item NEP
\item Processkø
\end{itemize}

\section{Forudsætninger} For at starte lagringen af et dataemne, skal
dataemnet være valideret, og et et passende tidsestimat skal være
udregnet. Samtidig skal det metadata der er identificeret og isoleret
i valideringsfasen være tilgængelig.
Processkøen skal være oppe og
tilgængelig, hvilket også gælder Fedora repositoriet.

\section{Initiering} Lagringsdelen af datadocken igangsættes efter
valideringen og estimeringen af dataemnet. Dataemnet samt metadata
bliver givet til lagringsdelen, med en pegepind til det fedora
repositorie data skal gemmes i, samt en peger til processkøen.

\section{Beskrivelse}
Efter validering og estimering skal data og metadata gemmes i repositoriet, og en tilhørende filpger placeres på processkøen.
Herunder er en liste med de nødvendige handlinger der foregår i lagringsrutinen.

\begin{itemize} 

\item Tjek adgang til Fedora repositoriet. Smider [RepositoryConnectionException] 
\item Tjek adgang til processkøen. Smider [QueueConnectionException] 

\item data indlæses i fedora repositoriet. Efter endt indlæsning returnerer Fedora en unik id (uuid), der peger på dataemnet. Smider [RepositoryConnectionException]

\item  metadata indlæses i repositoriet, der tilknyttes dataemnet ved
hjælp af den uuid der blev givet ved indlæsning af data. Smider
[RepositoryConnectionException]
\item 
Det returnerede uuid, lægges på processkøen.
Smider [QueueConnectionException] 

\end{itemize}

\section{Terminationstilstande}

\subsection{Succes}
Når dataemnet er indlæst i repositoriet og filpegeren ar lagt på køen,
er lagringsfasen overstået, og komponentet er klar til et nyt job.

\subsection{Exception}

\begin{itemize} 

\item \textbf{[RepositoryConnectionException]} 

Denne exception kan smides tre forskellige steder i
lagringsrutinen. Første sted den kan smides, er hvis der ikke kan
skabes forbindelse til repositoriet. andet sted er ved indlæsning af
data, hvor excptionen bliver smidt ved fejl under indlæsningen. Sidste
Sted den kan smides er ved en tilsvarende fejl ved indlæsning af
metadata.

\item \textbf{[QueueConnectionException]}

Denne exception kan smides to forskellige steder i
lagringsrutinen. Første gang denne exception kan kastes, er hvis der
ikke kan skabes forbindelse til processkøen. Anden gang den kan
smides, er når uuid'en lægges på processkøen.

\end{itemize}

\section{Status på use case}

Opgaven her er veldefineret, så der opstår nok ingen overaskelser her.

\end{document}
