\documentclass{article}
\usepackage{a4wide}
\usepackage[utf8]{inputenc} %             be able to use danish letters
\usepackage[danish]{babel} %                and danish macros
\usepackage{url}                            %, hyperref} hyperref makes clickable refs.
\usepackage{amsmath}
\renewcommand\sfdefault{phv}%               use helvetica for sans serif
\renewcommand\familydefault{\sfdefault}%    use sans serif by default
\setlength{\parskip}{2mm plus1mm minus1mm}% a bit of (rubber) space between paragraphs

\author{Lars Vensild Hørnell \and Søren Mollerup \and Steen
  Manniche\thanks{\{lvh, shm, stm\}\@dbc.dk}}
\date{\today}
\title{Usecase: \thanks{Dette dokument kan hentes fra \texttt{svn://svn.dbc.dk/OpenSearch/project\_information}}}

\begin{document}

\maketitle

\newpage

\tableofcontents
% Det er hensigten, at alle informationer der er i < eller > tegn skal
% slettes inden dokumentet bygges

\section{Use case beskrivelse}
<Overordnet beskrivelse af use casen, dens formål og resultater>


\section{Type}
<Konkret|Abstrakt>


\section{Relationer}
<Om use casen f.eks. anvender andre use cases>


\section{Aktører}
<Liste af de aktører der indgår>


\section{Forudsætninger}
<Typisk ting aktørerne skal have gjort, en tilstand systemet skal være
i, eller noget der på anden måde skal være tilgængeligt>


\section{Initiering}
<Hvorfor og hvordan use casen startes, som regel af en aktør>


\section{Beskrivelse}
<En beskrivelse af forløbet i use casen, evt. afbrudt af [exception] klammer>


\section{Terminationstilstande}

\subsection{Succes}
<Beskrivelse af en eller flere succes situationer>

\subsection{Exception}
<Beskrivelse af de undtagelser der kan forekomme, hvor use casen afbrydes>

\section{Status på use case}


<En beskrivelse af evt. ting der mangler afklaring omkring use casen>

\end{document}
