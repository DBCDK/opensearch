\documentclass{article}
\usepackage{a4wide}
\usepackage[utf8]{inputenc} %             be able to use danish letters
\usepackage[danish]{babel} %                and danish macros
\usepackage{url}                            %, hyperref} hyperref makes clickable refs.
\usepackage{amsmath}
\renewcommand\sfdefault{phv}%               use helvetica for sans serif
\renewcommand\familydefault{\sfdefault}%    use sans serif by default
\setlength{\parskip}{2mm plus1mm minus1mm}% a bit of (rubber) space between paragraphs

\author{\thanks{email}}
\date{\today}
\title{Usecase: Foresørgsel på Handlers i Transformation Engine\thanks{Dette dokument kan hentes fra \texttt{svn://svn.dbc.dk/OpenSearch/project\_information}}}

\begin{document}

\maketitle

\newpage

\tableofcontents

\section{Use case beskrivelse}

Denne usecase beskriver den situation, hvor der fra en klients
(muligvis et komponent der agerer klient) side bliver forespurgt på,
hvilke handlers der er registreret i \texttt{TransformationEngine}.

Handlere bliver typisk registreret ved opstarten af
\texttt{TransformationEngine} med en mimetype, et navn og selve
handleren.

Resultatet af udførelsen af en foresørgsel er en liste af
registrerede handlers%, enten totalt eller
for en given mimetype. Hvis kun en handler for en given mimetype eller
mimetype/navn findes, vil listen kun indeholde et element. Et negativt
svar kommer i form af en exception.  Mimetyperne skal overholde
standarden for mimetyper. Eksempler på ofte brugte mimetyper kan
findes i specificeret i /etc/mime.types eller på
\url{http://www.iana.org/assignments/media-types/index.html}.

\section{Type}
Konkret

\section{Relationer}
Umiddelbart ikke

\section{Aktører}
\begin{itemize}
\item evt. Systemadministrationsklienter
\item Opstartsproceduren for \texttt{TransformationEngine}
\item \texttt{DataDock}
\end{itemize}

\section{Forudsætninger}
Da en foresørgsel kan udføres enten i en opstartsfase - dvs. i forbindelse med registreringen af Handlers til \texttt{TransformationEngine} - eller umiddelbart før \texttt{TransformationEngine} går i gang med at Transformere givent data; i \texttt{DataDock}en, vil der kunne eksistere forskellige forudsætninger for gennemførelsen af denne usecase. 

\begin{itemize}
\item Der skal være en mimetype der kan foresørges på. Denne leveres i det metadata der følger med indleveringen af data i \texttt{DataDock}en.
\item \texttt{TransformationEngine} skal være startet.
\end{itemize}

\section{Initiering}
En foresørgsel bliver foretaget med en mimetype eller med en mimetype og et navn. Den primære nøgle for opslaget er mimetypen. Der kan være registreret flere Handlers for en given mimetype, hvorfor en foresørgsel også kan foretages med en mimetype og et navn på Handleren.

\section{Beskrivelse}

\begin{enumerate}
\item Klienten sender mimetype og evt navn til \texttt{TransformationEngine}
\item Mimetype checkes for at overholde accepterede mimetyper
\item [UknownMimetypeException] kastes hvis ikke mimetypen kendes
\item Der foretages opslag i handler-mappet
\item Der returneres navnet på handleren, en liste af navne på handlers
\item [UnknownHandlerException] kastes hvis ikke handleren findes
\end{enumerate}

\section{Terminationstilstande}

\subsection{Succes}
\begin{itemize}
\item Der returneres navn på en eller flere Handlers
\end{itemize}

\subsection{Exception}
\begin{itemize}
\item UnknownMimetypeException hvis mimetypen ikke accepteres af \texttt{TransformationEngine}
\item UnknownHandlerException hvis ikke handleren findes
\end{itemize}

Det kan være værd at overveje om der også skal findes et register for accepterede mimetyper, eller om dette skal specificeres vha. konfigurationer. Alternativt kan alle mimetyper i /etc/mime.types accepteres som gyldige nøgler for Handlers.

\section{Status på use case}
Der mangler afklaring på, hvilke exceptions der skal bruges i undtagelsestilstandende.

Der mangler også afklaring på om det skal være muligt at returnere alle registrerede handlers.




\end{document}
