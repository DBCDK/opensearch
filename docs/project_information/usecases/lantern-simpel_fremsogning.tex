\documentclass{article}
\usepackage{a4wide}
\usepackage[utf8]{inputenc} %             be able to use danish letters
\usepackage[danish]{babel} %                and danish macros
\usepackage{url}                            %, hyperref} hyperref makes clickable refs.
\usepackage{amsmath}
\renewcommand\sfdefault{phv}%               use helvetica for sans serif
\renewcommand\familydefault{\sfdefault}%    use sans serif by default
\setlength{\parskip}{2mm plus1mm minus1mm}% a bit of (rubber) space between paragraphs

\author{Lars Vensild Hørnell \and Søren Mollerup \and Steen
  Manniche\thanks{\{lvh, shm, stm\}\@dbc.dk}}
\date{\today}
\title{Use case: Lantern, simpel fremsøgnig ved søgning på termer
  \thanks{Dette dokument kan hentes fra \texttt{svn://svn.dbc.dk/OpenSearch/project\_information}}}

\begin{document}

\maketitle

\newpage

\tableofcontents
% Det er hensigten, at alle informationer der er i < eller > tegn skal
% slettes inden dokumentet bygges

\section{Use case beskrivelse}
Denne use case beskriver hvorledes Lantern komponenten modtager
termer og finder information ud fra dem og sender dette tilbage. Der
beskrives også hvordan Lantern modtager en forespørgsel på et
objekt og sender dette tilbage til spørgeren. 

\section{Type}
Konkret

\section{Relationer}


\section{Aktører}

\begin{itemize}
\item Lantern
\item NEP
\item Compass/Lucene
\item Storage
\item Fedora
\end{itemize}

\section{Forudsætninger}

\section{Initiering}
NEP sender en fremsøgningsforespørgsel til Lantern

\section{Beskrivelse}
Lantern modtager en forespørgsel fra NEP. Valideringsrutinen
forsøger at validerer forespørgslen. Hvis den ikke validerer sendes en
fejl [NotValidQueryException] til NEP. Hvis den validerer ser Lantern hvilken type
forespørgsel der er tale om. De to muligheder er:
\begin{enumerate}
\item Forespørgsel på en søgning udfra simple termer
\item Forespørgsel på et eller flere specifikke dataobjekter
\end{enumerate}

Hvis der er tale om den første type forespørgsel gøres følgende:
\begin{itemize}
\item Compass/Lucene sættes til at søge i indexerne i Storage med søgetermerne
  som argument (skal søgetermerne gennem noget behandling, fx en
  Analyzer, for at stemme overens med formatet af indholdet af indexerne?) 
\item Modtager søgeresultatet fra Compass/Lucene
\item Processerer søgeresultatet således at det stemmer overens med
  formatinformation i forespørgslen
\item Sender det processerede søgeresultat tilbage til NEP
\end{itemize}

Hvis der er tale om den anden type forespørgsel gøres følgende:
\begin{itemize}
\item Lantern henter de(t) ønskede dataobjekt(er) ud af Fedora
  udfra uuid'et.
\item Hvis uuid'erne ikke giver noget resultat smides
  [NotValidUuidException] og en fejlmeddelse sendes til NEP
\item De hentede objekter pakkes ind og sendes til NEP
\end{itemize}

\section{Terminationstilstande}

\subsection{Succes}
Lantern har enten afleveret en mængde søgeresultater til NEP
eller et eller flere dataobjekter.

\subsection{Exception}
\begin{itemize}
\item \texttt{[NotValidQueryException]} smides hvis en forespørgsel
  fra NEP ikke validerer.
\item \textt{[NotValidUuidException]} smides hvis uuid'et der forsøges
  hentes et objekt i Fedora på baggrund af ikke giver noget resultat
\end{itemize}
\section{Status på use case}

\begin{itemize}
\item Det skal afklares hvordan termerne skal formateres før de gives
  til Compas/Lucene 
\item Denne use case skal splittes op i flere mere præcise use cases
\end{itemize}

\end{document}
